\clearpage
\section{Introduction}

The International Model for Policy Analysis of Agricultural Commodities and Trade (IMPACT) was developed at the International Food Policy Research Institute (IFPRI) at the beginning of the 1990s to address a lack of long-term vision and consensus among policymakers and researchers about the actions necessary to feed the world in the future, reduce poverty, and protect the natural resource base. Over time, this economic model has been expanded and improved, and IMPACT is now a system of linked models around a core multimarket economic model of global production, trade, demand, and prices for agricultural commodities. This document updates and replaces previous technical reports that served as model documentation for IMPACT, in particular \cite{robinson2015international}.

The multimarket model simulates the operation of national and international markets, solving for production, demand, and prices that equate supply and demand across the globe. The core model is linked to a number of “modules” that include climate models (Earth System Models, ESMs), water models (hydrology, water basin management, and water stress models), crop simulation models like Decision Support System for Agrotechnology Transfer (DSSAT), value chain models (for example, sugar, oils, livestock), land use (pixel-level land use, cropping patterns by regions), nutrition and health models. The IMPACT model system integrates information flows among the component modules in a consistent equilibrium framework that supports longer-term scenario analysis.

Some of the model communication is one way, with no feedback links (for example, climate scenarios to hydrology models and crop simulation models), while other links require capturing feedback loops (for example, water demand from the core multimarket model and water supply from the water models must be reconciled to estimate water stress impacts on crop yields). The IMPACT model is designed for scenario analysis rather than forecasting—a distinction discussed in more detail in Section 3. It is a “structural” model in the sense that it simulates the operation of commodity markets and the behavior of economic “agents” (for example, producers and consumers) that determine supply and demand for agricultural commodities in those markets. In particular, it provides a detailed specification of production technology and shocks affecting productivity (for example, water shortages and changes in temperature). It is a partial equilibrium model in that it deals only with agricultural commodities and so covers only part of overall economic activity. Computable General Equilibrium (CGE) models, another class of long-run simulation models, cover the entire economy and hence are “complete” in the sense that they specify all economic flows and include all commodity markets and usually all factor markets (for example, labor and capital markets). The two types of models have different strengths and weaknesses for scenario analysis and have proven to be complementary in analysis of long-run trends under climate change.

Given its modular structure, the IMPACT model supports integrated analysis of the implications of physical, biophysical, and socioeconomic trends and phenomena, allowing for varied and in-depth analysis on a variety of key issues of interest to policymakers. As a flexible policy analysis tool, IMPACT has been used to research linkages between agriculture production and food security at the national and regional levels. IMPACT also has been used in commodity-level analyses and has contributed to thematic and interdisciplinary scenario-based projects.

The core multimarket model focuses on national and global markets including 158 countries. Agricultural production is specified by models of land supply, allocation of land to irrigated and rainfed crops, and determination of yields. Production is modeled at a sub-national level, including 320 regions called food production units (FPUs). FPUs correspond to water basins within national boundaries of these 158 countries (see Fig. \ref{fig:geography} for a geography overview and Appendix A for more detailed IMPACT geography). The multimarket model simulates 62 agricultural commodity markets, representing the bulk of food and cash crops (see Appendix B for a full list of commodities). 

\begin{figure}[hbt!]
    \centering
    \includegraphics[width=1\textwidth]{plots/FPUs.png}
    \caption{Spatial scale of IMPACT model's operation}
    \label{fig:geography}
\end{figure}

Fig. \ref{fig:framework} summarizes the links between major component modules and the core multimarket model, with arrows indicating information flows. The climate models (ESMs) provide climate data (temperature and precipitation) as inputs to the water and crop simulation models. Macroeconomic trends reflect projections from demographic and economic growth models. These links are one way, from these models to the multimarket and water models. The water models are dynamically linked to the multimarket model, with two-way flows of information over time. Other modules (for example, value chains, land allocation to crops) are integrated within periods with the core multimarket model. Finally, a set of post- solution modules calculates the results from scenario solution, with one-way communication from the multimarket model. A detailed schematic of the multimarket core model as well as a more detailed description of the integration of different modules within the IMPACT system can be seen in Section 5.

\begin{figure}[hbt!]
  \centering
  \includegraphics[width=\textwidth]{plots/2021.impact-diagram.v3.png}
  \caption{IMPACT modeling framework with their interactions}
  \label{fig:framework}
\end{figure}