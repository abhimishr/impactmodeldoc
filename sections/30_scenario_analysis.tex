\clearpage
\section{Scenario analysis}
\label{sec:scenanal}

Ex ante analysis of global agricultural markets several decades into the future requires a flexible, scenario-based approach that involves specification of the impacts of long-run drivers (such as changes in population, income, consumer behavior, climate, and technology development) whose nature is highly uncertain. Scenario analysis is a powerful analytical tool that allows policymakers to explore plausible futures in a systematic manner, considering future uncertainties. Scenario analysis is distinct from forecasting analysis in that the objective is not to predict the most likely outcome (usually extrapolating from historical experience). Instead, scenario analysis focuses on system dynamics, generating logically consistent future pathways that include trends and nonlinear interactions that may deviate significantly from past experience. Fig. \ref{fig:forceasting} illustrates the difference in the range of possibilities that are considered in scenario analysis versus traditional forecasting.

\begin{figure}[hbt!]
  \centering
  \includegraphics[width=1\textwidth]{plots/forecasting_vs_scenarios.png}
  \caption[Forecasting vs. scenario analysis]{Forecasting vs. scenario analysis. Source: Authors, using \cite{vervoort2013future}.}
  \label{fig:forceasting}
\end{figure}

Scenario analysis with simulation models allows policymakers to explore the robustness of
different policies by testing them against alternative futures focused on key uncertainties (for example, different climates). Scenario analysis is also an appropriate approach for exploring the effects of extreme events, whose probability may be low at any particular point in time but whose effects can be catastrophic \parencite{maackscenario}.

Simulation models like IMPACT are designed explicitly for scenario analysis and provide powerful tools for developing scenarios. They include strong logical, consistent constructs around which scenario narratives can be built, providing continuous checks for internal consistency of the scenario’s logic. They also allow the scenarios to be quantified and then simulated, permitting them to be tested and refined. The empirical results from simulating these scenarios then can give policymakers information not only about the direction of change but also about the magnitude of change suggested by the scenarios. These scenario results can be useful for informing policy decisions as well as providing many global and regional contexts for more detailed scenario development.

The IMPACT model supports analysis of a variety of alternative scenarios within the global agricultural economy. IMPACT has been used extensively for analyzing the effects of changes in socioeconomic trends, the environment, and technology. It is also designed to consider scenarios of changes in public investment patterns and trade policy. IMPACT specifically allows for analyzing alternative scenarios about how population, income, climate, and technologies may change over time. Borrowing from the scenario analysis literature, we can group these traditional scenarios into four categories: socioeconomic, environmental, political, and technological. This framework is similar to the one for identifying environmental forces proposed by Ian \cite{wilson1998mental} and is summarized in Table \ref{tab:framework_application}.

% Table generated by Excel2LaTeX from sheet 'socecon_etc'
\begin{table}[h]
  \centering
  \caption[Socioeconomic, environmental, political, and technological framework applied in IMPACT scenarios]{The socioeconomic, environmental, political, and technological framework applied in IMPACT scenarios. Source: Authors, with adaptation from \cite{wilson1998mental}.}
    \resizebox{\textwidth}{!}{
    \begin{tabular}{p{6.415em}rp{32.085em}}
    \toprule
    \multicolumn{1}{l}{Domain} &       & Examples in IMPACT \\
    \midrule
    \multirow{7}[2]{*}{Socioeconomic} &       & Population growth \\
    \multicolumn{1}{l}{} &       & Urban-rural households \\
    \multicolumn{1}{l}{} &       & Gross domestic product and economic development \\
    \multicolumn{1}{l}{} &       & Income distribution across households \\
    \multicolumn{1}{l}{} &       & Consumer behaviour \\
    \multicolumn{1}{l}{} &       & Price transmission and exchange rates \\
    \multicolumn{1}{l}{} &       & Input (fertilizers, pesticides, energy etc.) \\
    \midrule
    \multirow{2}[2]{*}{Environmental} &       & Water and land availability \\
    \multicolumn{1}{l}{} &       & Climate change \\
    \multicolumn{1}{l}{} &       & Pests and diseases \\
    \midrule
    \multirow{2}[2]{*}{Policy} &       & Investment in agriculture research and development \\
    \multicolumn{1}{l}{} &       & Trade policies (taxes, tariffs, consumer \& producer support policies) \\
    \midrule
    \multirow{2}[2]{*}{Infrastructure} &       & Investment in irrigation expansion and irrigation technologies \\
    \multicolumn{1}{l}{} &       & Investment in connectivity infrastructure for transportation \\
    \midrule
    Technological &       & Change in agricultural productivity due to improved genetics and management \\
    \bottomrule
    \end{tabular}%
    }%end resize
  \label{tab:framework_application}%
\end{table}%

IMPACT has been used extensively in developing and simulating regional as well as global scenarios. The core set of scenarios for which IMPACT is calibrated is global and has evolved over time as new topics of concern have arisen. For example, the potential effects of uncertain climate change have become a major issue of interest globally, and the most recent core scenarios for both IMPACT 2 and IMPACT 3 have been based on scenarios from the global community. For IMPACT 2, these were based on the Millennium Ecosystem Assessment \parencite{sarukhan2005millenium} and the Intergovernmental Panel on Climate Change’s (IPCC’s) fourth assessment report (AR4). For IMPACT 3, they are based on IPCC’s fifth assessment report (AR5). For IMPACT 4, they are based in IPCC's sixth assessment report (AR6).

Structural simulation models like IMPACT and global CGE models that focus on long-run scenario analysis are inherently difficult to validate. Validation for short-run forecasting models (for example, reduced-form, macro-econometric models) is easier in principle, involving simulating the model for recent years for which data are available (back-casting) and doing statistical analysis of the quality of the results. In econometric models, estimation and validation often go together—model parameters are estimated to maximize a measure of goodness of fit of the model to the data used in estimation. 

For long-run structural models, however, this approach is essentially infeasible. Structural simulation models involve many parameters and functional forms that are hard to estimate econometrically, and the models are designed to be used for scenario analysis that is often outside the domain of historical data. In this situation, validation necessarily involves (1) evaluating the validity of the structural design of the model, (2) assessing the quality of estimates of parameters using a variety of data sources, and (3) testing model projections with historical data when feasible. Testing with historical data is difficult since structural simulation models solve for long-run trends, while historical data often include shocks that are not part of the model design (for example, business cycle shocks that are not specified in a long-run trend model.

Validation of any model also must include a specification of the domain of applicability of the model—the universe for which the model can be applied. For an econometric model, the domain of applicability is essentially provided by the dataset used in model estimation. It is well understood that extrapolation of an econometric model outside the domain of its estimation dataset must be done with great care. For structural simulation models, specifying the domain of applicability is a major part of the model design and provides the starting point for model specification. Validation of a simulation model must reflect its intended domain of applicability.