\clearpage
\section{Summary}

Work on the IMPACT model began about 30 years ago in response to the need to look at long-run issues related to poverty alleviation, rural development, and food security. This need for tools to do long-run scenario analysis has only grown over time, with new challenges to the global food system like climate change coming to the forefront. This growing demand combined with improvements in computer hardware and software (methods) have spurred model growth and improvement to address new and ever-more-complex questions.  In response to these growing demands IMPACT’s domain of applicability has grown significantly. IMPACT 3.6 builds on the work of previous versions while consolidating these changes in a more flexible design that borrows from the best practices in the modeling literature.

These changes have been made to make IMPACT more flexible for future additions and improvements while at the same time making the model more transparent and accessible to a broader community of users. The benefits of these improvements have already proven beneficial, allowing a relatively small modeling team to better incorporate model feedback and new data and expert opinion. IMPACT is continually being improved as we incorporate new data and expertise to allow the model to be used in new and more complex ways. Currently a series of parallel efforts is being pursued to expand and improve on IMPACT and will become a part of future versions of IMPACT. Some of these improvements will lead to new modules and others to improvements to current modules. Table \ref{tab:future} summarizes the current IMPACT improvements that are in the pipeline.

% Table generated by Excel2LaTeX from sheet 'future'
\begin{table}[htbp]
  \centering
  \caption[Summary of ongoing IMPACT developments]{Summary of ongoing IMPACT developments. CIMSANS = Center for Integrated Modeling of Sustainable Agriculture and Nutrition Security; CSIRO = Commonwealth Scientific and Industrial Research Organization; GLOBE = Global CGE model ILRI = International Livestock Research Institute; IMPACT = International Model for Policy Analysis of Agricultural Commodities and Trade.\textcolor{red}{\textbf{tbd: make proper update, add collaborators etc.}}}
    \resizebox{\textwidth}{!}{
    \begin{tabular}{p{15.835em}p{37.665em}p{6.585em}}
    \toprule
    \textbf{Model improvement} & \textbf{Summary} & \textbf{Collaborators} \\
    \midrule
    Aquaculture module & The aquaculture module was dropped in the transition from IMPACT 1 to IMPACT 2. An exogenous module for fish was developed based on IMPACT 2 (Msangi et al 2013), and there is ongoing work to extend this module and fully integrate it into IMPACT 3.6+. & WorldFish \\
    \midrule
    Integrating GLOBE with IMPACT & GLOBE is a global CGE model. Many issues related to poverty alleviation and welfare are difficult to answer when only focusing on the agriculture sector. To be able to assess welfare effects as well as important interactions between the agriculture sector and other sectors of the economy, GLOBE is being calibrated and linked to IMPACT. & Institute of Development Studies \\
    \midrule
    Livestock module & Update the current handling of livestock in IMPACT to better reflect livestock production systems around the world. In addition incorporate more detailed handling of livestock diets and the direct and indirect effects of climate change. & ILRI, CSIRO \\
    \midrule
    Nutrition and health & Questions about nutrition are critical if we are to analyze food security. IMPACT 3 currently has modules that can assess trends on changes in undernourishment. These measures are being expanded to include new data, to look at nutrient deficiency, obesity, and updating the current food security module. In addition, we are linking IMPACT food demand to health modules to look at the changes diets may have on noncommunicable diseases. & CIMSANS, CSIRO, and Oxford University \\
    \bottomrule
    \end{tabular}
    }%
  \label{tab:future}%
\end{table}%
