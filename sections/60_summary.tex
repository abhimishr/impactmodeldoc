\clearpage
\section{Summary}

IFPRI began work on the IMPACT model about 30 years ago in response to the need to look at long-run issues related to poverty alleviation, rural development, and food security. This need for tools to do long-run scenario analysis has only grown over time, with new challenges to the global food system like climate change coming to the forefront. This growing demand combined with improvements in computer hardware and software (methods) have spurred model growth and improvement to address new and ever-more-complex questions.  In response to these growing demands IMPACT’s domain of applicability has grown significantly. IMPACT 3.6 builds on the work of previous versions while consolidating these changes in a more flexible design that borrows from the best practices in the modeling literature.

These changes have been made IMPACT more flexible for future additions and improvements while at the same time making the model more transparent and accessible to a broader community of users. These improvements have already proven beneficial, allowing a relatively small modeling team to better incorporate model feedback and new data and expert opinion. IMPACT is continually being improved as we incorporate new data and expertise to allow the model to be used in new and more complex ways. Currently a series of parallel efforts, including improvements to current modules and development of new modules, is being pursued to expand and improve on IMPACT and will become a part of future versions of IMPACT. 