\clearpage
\section*{International Food Policy Research Institute}

{\normalsize
The International Food Policy Research Institute (IFPRI), a research center of CGIAR, provides
research-based policy solutions to sustainably reduce poverty and end hunger and malnutrition in low- and middle-income countries. IFPRI was established in 1975 to identify and analyze alternative national and international strategies and policies for meeting the food needs of the developing world, with particular emphasis on low-income countries and on the poorer groups in those countries. Partnerships, communications, capacity strengthening, and data and knowledge management are essential components for translating IFPRI’s research to action and impact. The Institute’s regional and country programs play a critical role in responding to demand for food policy research and in delivering holistic support to country-led development. IFPRI collaborates with partners around the world. 

\vspace{0.01cm}
\href{https:\\www.ifpri.org}{www.ifpri.org}
}

\section*{CGIAR}

{\normalsize
CGIAR is a global research partnership for a food-secure future, dedicated to transforming food, land, and water systems in a climate crisis. CGIAR science aims to reduce poverty, enhance food and nutrition security, and improve natural resources and ecosystem services. As the world’s largest agricultural innovation network, its research is carried out by 13 \href{https://www.cgiar.org/our-centers-across-world}{CGIAR Centers} working around the world in close collaboration with hundreds of partners, including national and regional research institutes, civil society organizations, academia, development organizations, and the private sector.

\vspace{0.01cm}
\href{https:\\www.cgiar.org}{www.cgiar.org}
}
\thispagestyle{empty}% Reset page style to 'empty"