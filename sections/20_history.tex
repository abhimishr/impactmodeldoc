\clearpage
\section{History of the IMPACT model}

IMPACT was developed at IFPRI at the beginning of the 1990s to do medium to long term scenario analysis. In 1993, IFPRI launched the 2020 Vision for Food, Agriculture, and the Environment Initiative, and in 1995 the first results using IMPACT were published as a 2020 Vision discussion paper: \textit{Global Food Projections to 2020: Implications for Investment} \parencite{rosegrant1995global}, which analyzed the effects of population, investment, and trade scenarios on food security and nutrition status, especially in developing countries.

IMPACT continues to serve as the basis for research at IFPRI, examining the links between production of key food commodities and food security at the national level in the context of scenarios of future trends, including climate change. Studies focus on regional issues, commodity-level analyses, and crosscutting thematic issues. New developments in computational and modeling capacity, as well as new thematic questions, have spurred development of the IMPACT model system to ensure that it remains a relevant policy analysis tool.

Since 1995 IMPACT has gone through a process of constant expansion and improvement. New components and modules have been added, expanding the domain of applicability of the model system (Table \ref{tab:history} summarizes the major model developments over time). First, water and aquaculture were added in the first half of the 2000s. The full integration of an explicit water module in the modeling framework, in particular, was critical and was a focus of several IMPACT studies investigating the long-term dynamics of how water demand and availability would affect future food production, demand, and trade. The water model consists of three separate modules: (1) a global hydrology model, (2) water basin management models, and (3) water stress models that determine the impact of changes in water supply on crop yields. Later, links were added to food security modules to estimate changes in the number of undernourished children and crop models to allow for systematic analysis of climate change impacts on agriculture productivity and changes in food security. IMPACT 3 is now adding agricultural land markets, linking with land-use models to better handle competing demands for land and changes in greenhouse gas emissions due to land-use change in future analysis.

The latest version of IMPACT, i.e., v3.6 brings forward additional capabilities to the modeling framework. These include the latest climate impact projections from AR6 of IPCC, version 3 of the Shared Socioeconomic Pathways, ability to model planetary health diets, and, inclusion of post harves losses in our food systems.

\begin{landscape}
% Table generated by Excel2LaTeX from sheet 'Sheet1'
\begin{table}[htbp]
  \centering
  \caption{IMPACT model versions over time}
    \resizebox{\linewidth}{!}{            
    \begin{tabular}{llrrlrp{6.415em}l}
    Version & Solver & \multicolumn{1}{l}{Scope} & \multicolumn{1}{l}{Commodities} & Time scope & \multicolumn{1}{l}{Linkages} & \multicolumn{1}{r}{\textit{changelog}} & \multicolumn{1}{p{11.835em}}{Model description} \\
    \midrule
    \multirow{13}[6]{*}{1 to 1.x} & \multirow{13}[6]{*}{Gauss-Seidel} & \multicolumn{1}{l}{35 countries} & \multicolumn{1}{l}{17 total} & \multirow{4}[2]{*}{1995 - 2020} & \multirow{4}[2]{*}{} & \multicolumn{1}{c}{\multirow{4}[2]{*}{}} & \multicolumn{1}{l}{\multirow{4}[2]{*}{Rosegrant, Agcaoili-Sombilla, and Perez (1995)}} \\
          &       &       & \multicolumn{1}{l}{8 crop} &       &       & \multicolumn{1}{c}{} &  \\
          &       &       & \multicolumn{1}{l}{6 livestock} &       &       & \multicolumn{1}{c}{} &  \\
          &       &       & \multicolumn{1}{l}{3 processed} &       &       & \multicolumn{1}{c}{} &  \\
\cmidrule{3-8}          &       & \multicolumn{1}{l}{36 countries} & \multicolumn{1}{l}{32 total} & \multirow{4}[2]{*}{2000 - 2025} & \multicolumn{1}{l}{Water} & \multicolumn{1}{c}{\multirow{4}[2]{*}{}} & \multicolumn{1}{l}{\multirow{4}[2]{*}{Rosegrant, Sulser. et al. (2005)}} \\
          &       & \multicolumn{1}{l}{69 FPUs} & \multicolumn{1}{l}{14 crop} &       & \multicolumn{1}{l}{Value chains} & \multicolumn{1}{c}{} &  \\
          &       &       & \multicolumn{1}{l}{6 livestock} &       & \multicolumn{1}{l}{(processing)} & \multicolumn{1}{c}{} &  \\
          &       &       & \multicolumn{1}{l}{2 processed} &       &       & \multicolumn{1}{c}{} &  \\
\cmidrule{3-8}          &       & \multicolumn{1}{l}{115 countries} & \multicolumn{1}{l}{44 total} & \multirow{5}[2]{*}{2000 - 2050} & \multicolumn{1}{l}{Water} & \multicolumn{1}{c}{\multirow{5}[2]{*}{}} & \multicolumn{1}{l}{\multirow{5}[2]{*}{Rosegrant et al. (2008)}} \\
          &       & \multicolumn{1}{l}{281 FPUs} & \multicolumn{1}{l}{23 crop} &       & \multicolumn{1}{l}{Crop} & \multicolumn{1}{c}{} &  \\
          &       &       & \multicolumn{1}{l}{6 livestock} &       & \multicolumn{1}{l}{Food security} & \multicolumn{1}{c}{} &  \\
          &       &       & \multicolumn{1}{l}{15 processed} &       & \multicolumn{1}{l}{Value chains} & \multicolumn{1}{c}{} &  \\
          &       &       &       &       & \multicolumn{1}{l}{(processing)} & \multicolumn{1}{c}{} &  \\
    \midrule
    \multirow{5}[2]{*}{2 to 2.x} & \multirow{5}[2]{*}{Path} & \multicolumn{1}{l}{115 countries} & \multicolumn{1}{l}{45 total} & \multirow{5}[2]{*}{2000 - 2050} & \multicolumn{1}{l}{Water} & \multicolumn{1}{c}{\multirow{5}[2]{*}{}} & \multicolumn{1}{l}{\multirow{5}[2]{*}{Rosegrant and IMPACT development team (2012)}} \\
          &       & \multicolumn{1}{l}{281 FPUs} & \multicolumn{1}{l}{24 crop} &       & \multicolumn{1}{l}{Crop} & \multicolumn{1}{c}{} &  \\
          &       &       & \multicolumn{1}{l}{6 livestock} &       & \multicolumn{1}{l}{Food security} & \multicolumn{1}{c}{} &  \\
          &       &       & \multicolumn{1}{l}{15 processed} &       & \multicolumn{1}{l}{Value chains} & \multicolumn{1}{c}{} &  \\
          &       &       &       &       & \multicolumn{1}{l}{(processing)} & \multicolumn{1}{c}{} &  \\
    \midrule
    \multirow{5}[2]{*}{3} & \multirow{5}[2]{*}{Path} & \multicolumn{1}{l}{158 countries} & \multicolumn{1}{l}{62 total} & \multirow{5}[2]{*}{2005 - 2050} & \multicolumn{1}{l}{Water} & \multicolumn{1}{c}{\multirow{5}[2]{*}{}} & \multicolumn{1}{l}{\multirow{5}[2]{*}{Robinson et al. 2015}} \\
          &       & \multicolumn{1}{l}{320 FPUs} & \multicolumn{1}{l}{39 crop} &       & \multicolumn{1}{l}{Crop} & \multicolumn{1}{c}{} &  \\
          &       &       & \multicolumn{1}{l}{6 livestock} &       & \multicolumn{1}{l}{Food security} & \multicolumn{1}{c}{} &  \\
          &       &       & \multicolumn{1}{l}{17 processed} &       & \multicolumn{1}{l}{Value chains} & \multicolumn{1}{c}{} &  \\
          &       &       &       &       & \multicolumn{1}{l}{(processing)} & \multicolumn{1}{c}{} &  \\
    \midrule
    \multirow{5}[2]{*}{3.6+} & \multirow{5}[2]{*}{Path} & \multicolumn{1}{l}{158 countries} & \multicolumn{1}{l}{62 total} & \multirow{5}[2]{*}{2020 - 2050} & \multicolumn{1}{l}{Water} & AR6 climate impacts & \multicolumn{1}{l}{\multirow{5}[2]{*}{Current paper}} \\
          &       & \multicolumn{1}{l}{320 FPUs} & \multicolumn{1}{l}{39 crop} &       & \multicolumn{1}{l}{Crop} & SSP version 3 &  \\
          &       &       & \multicolumn{1}{l}{6 livestock} &       & \multicolumn{1}{l}{Food security} & Planteary health diets &  \\
          &       &       & \multicolumn{1}{l}{17 processed} &       & \multicolumn{1}{p{6.415em}}{Value chains (processing)} & Post harvest losses &  \\
          &       &       &       &       &       & IPRs linked to GDP changes &  \\
    \bottomrule
    \end{tabular}%
    }%
  \label{tab:history}%
\end{table}%
\end{landscape}

Improving availability of data and greater computing capacity has allowed for increasing coverage of commodity markets, expanding from the original 17 commodities and 35 countries to the current 62 commodities (and growing) and 158 countries. The model has increased not only the breadth of coverage but also the depth of the commodity markets, with each subsequent version building on previous work to better model basic value chains. For example, the first version started with two aggregate processed commodities: food oils and oil meals. IMPACT 3 now simulates six oilseed complexes (groundnut, palm, rapeseed, soybean, sunflower, and other oilseeds). The model also includes the value chain for livestock, from feed grains to dressed meat and dairy. The IMPACT 3 model includes a general treatment of value chains that provides a flexible framework that will allow for the addition of future processing sectors.

This focus on increasing the breadth and depth of IMPACT’s modeling capacity has required significant data work. As part of the transition to IMPACT 3 and above, a new data management and estimation system was developed to handle the increased volume and complexity of data needed to support the model. This system is treated as a separate module that includes diagnostic tools to analyze and clean the data and estimation procedures to generate a consistent database. Appendix C summarizes the current sources of data used in IMPACT.

The core multimarket model and many of the linked modules are written in General Algebraic Modeling System (GAMS). The multimarket model code has gone through several major revisions moving from using a Gauss-Seidel solution method in IMPACT 1 to using sophisticated nonlinear solvers called GAMS that greatly improve solution robustness and speed. In addition, software design that incorporates best practices of modularity has become critical in the design of the latest version (IMPACT 3), laying the foundation for future model development. The user community of the IMPACT model, including model users and those interested in sharing scenario results, has grown significantly since 1995. In response to increased interest in sharing this tool with others to improve policymaking worldwide, IFPRI has held a series of IMPACT training workshops all over the world. The first publicly available version of IMPACT was released in 2005.  More recent versions of IMPACT have incorporated major developments in user interface to ease the use of IMPACT. 

Significant efforts also have been made in developing interfaces for running IMPACT as well as sophisticated data visualization tools to facilitate and encourage the use of IMPACT in policy analysis. One such visualization tool is App for IMPACT (ARIA - \textbf{A}pp fo\textbf{R} \textbf{I}mp\textbf{A}ct). This tool is developed in-house by the Foresight and Policy Modeling Unit at IFPRI and is made available as an open-source tool. Source code of ARIA is hosted on GitHub at \hyperlink{https://github.com/ifpri/aria}{github.com/ifpri/aria} and its documentation is available via IFPRI's R-universe at \hyperlink{https://ifpri.r-universe.dev/ARIA}{ifpri.r-universe.dev/ARIA} \parencite{Mishra_App_for_IMPACT}.